\chapter{Conclusiones} 

En este proyecto se ha implementado una biblioteca con 5 algoritmos 
de detección de anomalías basados en proximidad. Cada uno de ellos  
busca afrontar el problema y sus inconvenientes desde una perspectiva 
y técnica diferente. En el momento de esta publicación no existe  
ninguna biblioteca con todos estos algoritmos ya sea en Python o  
en otros lenguajes de programación. 
 

A nivel técnico, hemos visto como existen multitud de factores que  
pueden afectar al resultado final por lo que cada enfoque puede ser 
interesante según los datos a los que nos enfrentamos. También hemos 
visto como el problema de la detección de anomalías tiene una clara  
aplicación al mundo real y nuestros algoritmos pueden aplicarse a  
conjuntos de datos reales. 
 

En conclusión, hemos desarrollado una biblioteca o herramienta que cumple 
los requisitos que nos planeábamos al comienzo del estudio, buenos resultados 
con un pequeño tiempo de ejecución, estandarización, respuesta clara y seguridad.  
 
\section{Trabajo futuro} 
Como trabajos futuros se podrían seguir implementando más algoritmos 
para aumentar la paleta de posibilidades para aplicar ante los diferentes 
problemas que les pueden surgir a los usuarios finales de nuestra biblioteca. 

También podríamos ampliar el campo de técnicas incluidas e introducir  
algoritmos de otros enfoques como puede ser técnicas estadísticas, que  
aumentarían la utilidad y funcionalidad de nuestra biblioteca.



