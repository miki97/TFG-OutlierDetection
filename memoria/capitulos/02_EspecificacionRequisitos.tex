
\chapter{Especificacion de requisitos}
\section{Requistos funcionales}
Nuestro proyecto pretende desarrollar una biblioteca competitiva y de 
gran utilidad. Para ello vamos a necesitar cumplir los siguientes requisitos:

\begin{itemize}
    \item \textbf{Alta velocidad de procesamiento.} Debemos tener en cuenta
    que estamos en un campo donde si queremos tener una aplicación real de
    nuestra biblioteca, esta debe ser rápida y dar una respuesta en un tiempo
    razonablemente pequeño. El reto viene dado porque en nuestra área suele
    trabajarse con grandes volúmenes de datos, lo que supone que si no realizamos
    los procedimientos de manera eficiente, nuestra biblioteca puede demorarse más
    tiempo del deseado.

    \item \textbf{Estandarización Sklearn.} La biblioteca Sklearn \cite{APIReferenceScikitlearn} 
    se ha establecido como un estándar en Python en cuanto a algoritmos y métodos
    relacionados con el matching learning. Por tanto, si queremos que nuestra biblioteca
    pueda llegar al mayor número de usuarios y sea de un uso fácil e intuitivo, deberemos
    seguir el estándar de clase que se establece en Sklearn.

    \item \textbf{Respuesta clara.} Si realizamos un pequeño vistazo a los algoritmos
    más conocidos, podemos observar que la salida de cada algoritmo ofrece su propia
    puntuación o score. Por tanto, para hacer nuestra biblioteca más funcional se
    realizará una estandarización de los resultados de cada algoritmo, transformando 
    la salida en una probabilidad y permitiendo la comparación entre diferentes algoritmos.

    \item \textbf{Seguridad.} Si queremos que nuestra biblioteca sea referente y realmente
    útil, necesitamos la seguridad de que los resultados que nos ofrecen son los correctos
    y no exista duda de si nuestra biblioteca puede tener un comportamiento erróneo. Para ello
    será un requisito muy importante la realización de una gran cantidad de pruebas, con el fin 
    de ofrecer el producto más fiable.

    \item \textbf{Experimentación con datos reales.} Un requisito de nuestra biblioteca será
    demostrar que funciona para cualquier situación que se pueda presentar y para cualquier
    conjunto de datos reales. Por tanto, es necesario una experimentación con un gran número
    de conjuntos de datos reales que demuestren el buen comportamiento de nuestra biblioteca
    ante cualquier problema de la vida cotidiana.

\end{itemize}

\section{Objetivos}
En primer lugar, vamos a definir los objetivos que se busca alcanzar en este 
proyecto y sobre los cuales se a desarrollado el trabajo realizado.
\begin{itemize}
    \item \textbf{Revisión de la bibliografía existente.} En primer lugar
    se pretende explorar los algoritmos y propuestas ya publicadas, además de
    lectura de libros especializados en la materia. Esto nos permite
    adquirir un conocimiento profundo en la materia y un
    mejor desarrollo y entendimiento de los algoritmos implementados.
    Además, nos aportará una base de algoritmos para su posible implementación.

    \item \textbf{Estudio de cada algoritmo y diseño.} El segundo objetivo es un 
    estudio en profundidad de cada algoritmo para conocer los detalles de
    implementación. Se debe realizar un patrón de diseño para generar
    estructuras generales y reutilizables para todos ellos.

    \item \textbf{Implementación y prueba.} El tercer objetivo será
    realizar una implementación de cada algoritmo, basándose en la
    bibliografía y la realización de pruebas para confirmar el correcto
    funcionamiento de estos. Para ello se comprobará su correcto funcionamiento
    tanto en datos sintéticos como en datos reales.

    \item \textbf{Comparación y análisis.} El último objetivo es la
    realización de una comparación entre los algoritmos implementados y 
    analizar el comportamiento, las ventajas y desventajas que cada uno
    de estos. Para así determinar cuáles serán los casos más atractivos para
    la utilización de cada algoritmo. Se probará con una gran
    batería de conjuntos de datos y con los resultados se extraerán conclusiones.
\end{itemize}



