\chapter{Planificación}

\section{Planificación temporal}
El proyecto al cual nos enfrentamos tiene una dimensión considerable y
como hemos visto en los requisitos tenemos varios pasos complejos que
completar. Por tanto, la planificación debe ser fundamental. Cierto
es que el proceso de investigación está presente en todas las etapas,
pero en el siguiente diagrama veremos cómo se ha planificado el tiempo
para el desarrollo del proyecto

\begin{figure}[h]
\begin{ganttchart}[
    canvas/.append style={fill=none, draw=black!5, line width=.75pt},
    vgrid={*1{draw=black!5, line width=.75pt}},
    today=20,
    today label font=\scriptsize\scshape,
    title/.style={draw=none, fill=none},
    title label font=\scshape\footnotesize,
    title label node/.append style={below=7pt},
    include title in canvas=false,
    bar/.append style={draw=none, fill=black!63}
    ]{1}{20}


    \gantttitle{Sep}{2}
    \gantttitle{Oct}{2}
    \gantttitle{Nov}{2}
    \gantttitle{Dic}{2}
    \gantttitle{Ene}{2}
    \gantttitle{Feb}{2}
    \gantttitle{Mar}{2}
    \gantttitle{Abr}{2}
    \gantttitle{May}{2}
    \gantttitle{Jun}{2}
    \gantttitle{Jul}{2}\\

    \ganttbar{Investigación}{1}{6} \\
    \ganttbar{Análisis}{4}{8} \\
    \ganttbar{Diseño}{6}{8} \\
    \ganttbar{Implementación}{7}{18} \\
    \ganttbar{Pruebas}{11}{18} \\
    \ganttbar{Experimentación}{16}{20} \\
    \ganttbar{Memoria}{16}{20}

\end{ganttchart}
\caption{Planificación del proyecto}
\label{fig:gantt}
\end{figure}
En este diagrama de Gannt podemos ver que la planificación
era bastante clara. La primera fase era un proceso de investigación
y adquisición de una base de conocimientos que permitiera una comprensión
de los conceptos necesarios para el desarrollo del proyecto. Este proceso
conlleva un trabajo de inmersión en el campo a través de la lectura de libros
y artículos. El principal libro usado como referencia ha sido \cite{aggarwalOutlierAnalysis2017}.

A continuación, una etapa más de análisis y diseño de los algoritmos a implementar.
En esta etapa ya poseíamos una base de conocimientos fuertes para afrontar las especificaciones
de cada algoritmo. En esta etapa se analiza y se profundiza en los componentes
y desarrollo de cada algoritmo de nuestra biblioteca
Por último, tenemos la etapa de implementación, pruebas y experimentación, que 
en parte se han ido desarrollando en paralelo. En primer lugar, si existe una fase
de implementación del esquema general que siguen todos nuestros algoritmos y de la
base de partida. Después, en la implementación de cada algoritmo se prefirió probar
el funcionamiento tras la implementación de este. 

Como podemos ver existe un alto solapamiento entre las distintas etapas. Esto
se debe a que no se ha desarrollado con una metodología típicamente en cascada,
sino que se ha usado una metodología iterativa donde se iban realizando las 
fases de análisis, implementación y prueba de cada algoritmo hasta confirmar
su correcto funcionamiento. Una vez que se tenía un algoritmo listo se pasaba al
siguiente. De este modo podíamos profundizar mejor en la investigación y en los 
conceptos de cada algoritmo.

Por último, mencionar la etapa de desarrollo de la memoria donde se ha realizando
un trabajo de recopilación y de exposición de todo el trabajo que se ha realizado
durante este periodo de trabajo.


\section{Material necesario}
Para la implementación de este proyecto ha sido necesario un ordenador para el desarrollo.
En mi caso he utilizado mi ordenador personal, un Asus X550VX con un microprocesador de
Intel i5-6300HQ con cuatro núcleos físicos y 8 GB de RAM. Como podemos ver es un ordenador
no excesivamente potente y realmente con cualquier ordenador con un sistema Unix y un procesador
medianamente potente puede realizarse sin problema el desarrollo. Si debemos mencionar
que en cuanto a la memoria si han sido estrictamente necesarios los 8GB 
ya que manejamos grandes cantidades de datos y en algunos casos se ha alcanzado un uso de 
memoria de 7GB

También ha sido necesario una conexión a internet, recurso indispensable para acceder a la
documentación y paper de los algoritmos implementados. Además de proporcionar información
adicional ya sea en temas teóricos o de implementación. Este material es de fácil acceso y
en nuestro caso gratuito en la red de la universidad.



