\chapter*{}
%\thispagestyle{empty}
%\cleardoublepage

%\thispagestyle{empty}

\input{portada/portada_2}



\cleardoublepage
\thispagestyle{empty}

\begin{center}
{\large\bfseries \myTitle: \mySubTitle}\\
\end{center}
\begin{center}
\myName\\
\end{center}

%\vspace{0.7cm}
\noindent{\textbf{Palabras clave}: detección de anomalias, 
biblioteca python, aprendizaje no supervisado, clustering}\\

\vspace{0.7cm}
\noindent{\textbf{Resumen}}\\

En este trabajo se ha implementado una biblioteca en Python, para ofrecer
ayuda y resolución al problema de detección de anomalías. Para ello se han
implementado diversos algoritmos basados en la proximidad entre los distintos
ejemplos de datos. Se han implementado algoritmos de carácter más general y
tradicional, a la vez que se a buscado algoritmos innovadores y que profundizan 
más en los distintos inconvenientes que establece este reto. Con el fin de
analizar y comprobar la calidad de cada algoritmo, se ha desarrollado una 
experimentación con más de 20 conjuntos de datos. Con dicha experimentación,
buscaremos determinar las mejores condiciones para cada algoritmo.
\cleardoublepage


\thispagestyle{empty}


\begin{center}
{\large\bfseries 
Algorithm library for outlier detection: 
Algorithms based on proximity techniques}\\
\end{center}
\begin{center}
\myName\\
\end{center}

%\vspace{0.7cm}
\noindent{\textbf{Keywords}: outlier detection, Python library, unsupervised learning, 
}\\

\vspace{0.7cm}
\noindent{\textbf{Abstract}}\\

In this work, a Python library has been implemented, to offer help and resolution to the
anomaly detection problem. For this, diverse algorithms based on the proximity between the
different data examples have been implemented. we have implemented algorithms some classic
algorithms as well as some new algorithms that represent the state-of-the-art of the resolution
of the anomaly detetion problem. In order to analyze and
verify the quality of each algorithm, an experimentation with more than 20 data sets has been developed.
With this experimentation, we determine which are the best conditions for each algorithm.

\chapter*{}
\thispagestyle{empty}

\noindent\rule[-1ex]{\textwidth}{2pt}\\[4.5ex]

Yo, \textbf{Miguel Ángel López Robles}, alumno de la titulación Ingeniría Informática de la \textbf{Escuela Técnica Superior
de Ingenierías Informática y de Telecomunicación de la Universidad de Granada}, con DNI 76065425D, autorizo la
ubicación de la siguiente copia de mi Trabajo Fin de Grado en la biblioteca del centro para que pueda ser
consultada por las personas que lo deseen.

\vspace{6cm}

\noindent Fdo: Miguel Ángel López Robles

\vspace{2cm}

\begin{flushright}
Granada a 24 de junio de 2019 .
\end{flushright}


\chapter*{}
\thispagestyle{empty}

\noindent\rule[-1ex]{\textwidth}{2pt}\\[4.5ex]

D. \textbf{\myProf}, Profesor del Departamento de Ciencias de la Computación e Inteligencia Artificial  de la Universidad de Granada.

\vspace{0.5cm}

%D. \textbf{Nombre Apellido1 Apellido2 (tutor2)}, Profesor del Área de XXXX del Departamento YYYY de la Universidad de Granada.


\vspace{0.5cm}

\textbf{Informan:}

\vspace{0.5cm}

Que el presente trabajo, titulado \textit{\textbf{\myTitle, \mySubTitle}},
ha sido realizado bajo su supervisión por \textbf{\myName}, y autorizamos la defensa de dicho trabajo ante el tribunal
que corresponda.

\vspace{0.5cm}

Y para que conste, expiden y firman el presente informe en Granada a 24 de junio de 2019 .

\vspace{1cm}

\textbf{Los directores:}

\vspace{5cm}

\noindent \textbf{\myProf}% \ \ \ \ \ Nombre Apellido1 Apellido2 (tutor2)}

\chapter*{Agradecimientos}
\thispagestyle{empty}

       \vspace{1cm}



Muchas gracias a toda mi familia por el apoyo incondicional en todos estos años,
sin el cual no hubiera sido posible todos estos años de estudio. A mis profesores,
que tanto me han enseñado y que han ayudado a forjar la persona que soy. Por último
y más importante a Andrea que me apoyo cuando más lo necesitaba.
