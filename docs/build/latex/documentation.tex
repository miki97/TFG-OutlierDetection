%% Generated by Sphinx.
\def\sphinxdocclass{report}
\documentclass[letterpaper,10pt,spanish]{sphinxmanual}
\ifdefined\pdfpxdimen
   \let\sphinxpxdimen\pdfpxdimen\else\newdimen\sphinxpxdimen
\fi \sphinxpxdimen=.75bp\relax

\PassOptionsToPackage{warn}{textcomp}
\usepackage[utf8]{inputenc}
\ifdefined\DeclareUnicodeCharacter
% support both utf8 and utf8x syntaxes
  \ifdefined\DeclareUnicodeCharacterAsOptional
    \def\sphinxDUC#1{\DeclareUnicodeCharacter{"#1}}
  \else
    \let\sphinxDUC\DeclareUnicodeCharacter
  \fi
  \sphinxDUC{00A0}{\nobreakspace}
  \sphinxDUC{2500}{\sphinxunichar{2500}}
  \sphinxDUC{2502}{\sphinxunichar{2502}}
  \sphinxDUC{2514}{\sphinxunichar{2514}}
  \sphinxDUC{251C}{\sphinxunichar{251C}}
  \sphinxDUC{2572}{\textbackslash}
\fi
\usepackage{cmap}
\usepackage[T1]{fontenc}
\usepackage{amsmath,amssymb,amstext}
\usepackage{babel}



\usepackage{times}
\expandafter\ifx\csname T@LGR\endcsname\relax
\else
% LGR was declared as font encoding
  \substitutefont{LGR}{\rmdefault}{cmr}
  \substitutefont{LGR}{\sfdefault}{cmss}
  \substitutefont{LGR}{\ttdefault}{cmtt}
\fi
\expandafter\ifx\csname T@X2\endcsname\relax
  \expandafter\ifx\csname T@T2A\endcsname\relax
  \else
  % T2A was declared as font encoding
    \substitutefont{T2A}{\rmdefault}{cmr}
    \substitutefont{T2A}{\sfdefault}{cmss}
    \substitutefont{T2A}{\ttdefault}{cmtt}
  \fi
\else
% X2 was declared as font encoding
  \substitutefont{X2}{\rmdefault}{cmr}
  \substitutefont{X2}{\sfdefault}{cmss}
  \substitutefont{X2}{\ttdefault}{cmtt}
\fi


\usepackage[Sonny]{fncychap}
\ChNameVar{\Large\normalfont\sffamily}
\ChTitleVar{\Large\normalfont\sffamily}
\usepackage{sphinx}

\fvset{fontsize=\small}
\usepackage{geometry}

% Include hyperref last.
\usepackage{hyperref}
% Fix anchor placement for figures with captions.
\usepackage{hypcap}% it must be loaded after hyperref.
% Set up styles of URL: it should be placed after hyperref.
\urlstyle{same}
\addto\captionsspanish{\renewcommand{\contentsname}{Contents:}}

\usepackage{sphinxmessages}
\setcounter{tocdepth}{1}



\title{PyLOD Documentation}
\date{28 de agosto de 2019}
\release{1.0}
\author{Miguel Ángel López Robles}
\newcommand{\sphinxlogo}{\vbox{}}
\renewcommand{\releasename}{Versión}
\makeindex
\begin{document}

\ifdefined\shorthandoff
  \ifnum\catcode`\=\string=\active\shorthandoff{=}\fi
  \ifnum\catcode`\"=\active\shorthandoff{"}\fi
\fi

\pagestyle{empty}
\sphinxmaketitle
\pagestyle{plain}
\sphinxtableofcontents
\pagestyle{normal}
\phantomsection\label{\detokenize{index::doc}}



\chapter{Introduccion}
\label{\detokenize{index:introduccion}}
Bienvenido a PyDBOD, la biblioteca de Python para la detección de anomalías usando
algoritmos basados en distancias. En esta bibliotica tienes una amplia selección de
algoritmos los cuales vamos a documentar a continuación. El uso de todos se reduce a
la creación de un objeto de la clase respectiva y el uso del método \sphinxstylestrong{fit\_predict}.

Para instalar el paquete o obtener una distribución usar el repositorio en github o
en PyPI;

\sphinxurl{https://github.com/miki97/TFG-OutlierDetection}

\sphinxurl{https://pypi.org/project/PyDBOD/}


\chapter{LOF}
\label{\detokenize{index:lof}}
Local Outlier Factor (LOF), o en español factor de valor atıípico local, es
una cuantificación del valor atípico de un punto perteneciente al conjunto de
datos. Esta cuantificación es capaz de ajustar las variaciones en las densidades
locales.
\index{LOF() (función incorporada)@\spxentry{LOF()}\spxextra{función incorporada}}

\begin{fulllineitems}
\phantomsection\label{\detokenize{index:LOF}}\pysiglinewithargsret{\sphinxbfcode{\sphinxupquote{LOF}}}{\emph{k = 20}}{}
Constructor para la creación del objeto de la clase LOF.
\begin{quote}\begin{description}
\item[{Parámetros}] \leavevmode
\sphinxstyleliteralstrong{\sphinxupquote{int}} \textendash{} k, número de k vecinos a calcular

\item[{Tipo del valor devuelto}] \leavevmode
objeto de la clase LOF

\end{description}\end{quote}

\end{fulllineitems}

\index{fit\_predict() (función incorporada)@\spxentry{fit\_predict()}\spxextra{función incorporada}}

\begin{fulllineitems}
\phantomsection\label{\detokenize{index:fit_predict}}\pysiglinewithargsret{\sphinxbfcode{\sphinxupquote{fit\_predict}}}{\emph{data}}{}
Método para aplicar el algoritmo LOF a una matriz de datos.
\begin{quote}\begin{description}
\item[{Parámetros}] \leavevmode
\sphinxstyleliteralstrong{\sphinxupquote{numpy.array}} \textendash{} data, matriz de datos

\item[{Tipo del valor devuelto}] \leavevmode
numpy.array de puntuaciones de anomalía

\end{description}\end{quote}

\end{fulllineitems}



\chapter{LOOP}
\label{\detokenize{index:loop}}
Local Outlier Probability (LoOP), esta técnica combina varios conceptos.
En primer lugar, la idea de localidad, los algoritmos basados en densidad
como LOF. Por otro lado, LOCI
con conceptos probabilıísticos.
\index{LOOP() (función incorporada)@\spxentry{LOOP()}\spxextra{función incorporada}}

\begin{fulllineitems}
\phantomsection\label{\detokenize{index:LOOP}}\pysiglinewithargsret{\sphinxbfcode{\sphinxupquote{LOOP}}}{\emph{k = 20}, \emph{lamda=3}}{}
Constructor para la creación del objeto de la clase LOOP.
\begin{quote}\begin{description}
\item[{Parámetros}] \leavevmode\begin{itemize}
\item {} 
\sphinxstyleliteralstrong{\sphinxupquote{int}} \textendash{} k, número de k vecinos a calcular

\item {} 
\sphinxstyleliteralstrong{\sphinxupquote{int}} \textendash{} lamda, párametro para regular la normalización

\end{itemize}

\item[{Tipo del valor devuelto}] \leavevmode
objeto de la clase LOOP

\end{description}\end{quote}

\end{fulllineitems}

\index{fit\_predict() (función incorporada)@\spxentry{fit\_predict()}\spxextra{función incorporada}}

\begin{fulllineitems}
\pysiglinewithargsret{\sphinxbfcode{\sphinxupquote{fit\_predict}}}{\emph{data}}{}
Método para aplicar el algoritmo LOOP a una matriz de datos.
\begin{quote}\begin{description}
\item[{Parámetros}] \leavevmode
\sphinxstyleliteralstrong{\sphinxupquote{numpy.array}} \textendash{} data, matriz de datos

\item[{Tipo del valor devuelto}] \leavevmode
numpy.array de probabilidad anomalia {[}0-1{]}

\end{description}\end{quote}

\end{fulllineitems}



\chapter{LDOF}
\label{\detokenize{index:ldof}}
Local Outlier Probability (LoOP), utiliza la distancia relativa
de un objeto a sus vecinos para medir la cantidad de objetos que se desvıían
de su vecindario disperso.
\index{LDOF() (función incorporada)@\spxentry{LDOF()}\spxextra{función incorporada}}

\begin{fulllineitems}
\phantomsection\label{\detokenize{index:LDOF}}\pysiglinewithargsret{\sphinxbfcode{\sphinxupquote{LDOF}}}{\emph{k = 20}}{}
Constructor para la creación del objeto de la clase LDOF.
\begin{quote}\begin{description}
\item[{Parámetros}] \leavevmode
\sphinxstyleliteralstrong{\sphinxupquote{int}} \textendash{} k, número de k vecinos a calcular

\item[{Tipo del valor devuelto}] \leavevmode
objeto de la clase LOOP

\end{description}\end{quote}

\end{fulllineitems}

\index{fit\_predict() (función incorporada)@\spxentry{fit\_predict()}\spxextra{función incorporada}}

\begin{fulllineitems}
\pysiglinewithargsret{\sphinxbfcode{\sphinxupquote{fit\_predict}}}{\emph{data}}{}
Método para aplicar el algoritmo LDOF a una matriz de datos.
\begin{quote}\begin{description}
\item[{Parámetros}] \leavevmode
\sphinxstyleliteralstrong{\sphinxupquote{numpy.array}} \textendash{} data, matriz de datos

\item[{Tipo del valor devuelto}] \leavevmode
numpy.array de puntuaciones de anomalía

\end{description}\end{quote}

\end{fulllineitems}



\chapter{PINN-LOF}
\label{\detokenize{index:pinn-lof}}
Projection-Indexed Nearest-Neighbour (PINN), en este algoritmo se
propone un método de detección de valores atıípicos
locales proyectivo basado en LOF.


\begin{fulllineitems}
\pysigline{\sphinxbfcode{\sphinxupquote{PINN-LOF(k~=~20,~t=2,~s=1,~h=20)}}}
Constructor para la creación del objeto de la clase PINN-LOF.
\begin{quote}\begin{description}
\item[{Parámetros}] \leavevmode\begin{itemize}
\item {} 
\sphinxstyleliteralstrong{\sphinxupquote{int}} \textendash{} k, número de k vecinos a calcular

\item {} 
\sphinxstyleliteralstrong{\sphinxupquote{int}} \textendash{} t, probabilidad de seleccion de caracteristicas para la proyección

\item {} 
\sphinxstyleliteralstrong{\sphinxupquote{int}} \textendash{} s, probabilidad de selección para la proyección

\item {} 
\sphinxstyleliteralstrong{\sphinxupquote{int}} \textendash{} h, número de k vecinos a calcular en la proyección

\end{itemize}

\item[{Tipo del valor devuelto}] \leavevmode
objeto de la clase PINN-LOF

\end{description}\end{quote}

\end{fulllineitems}

\index{fit\_predict() (función incorporada)@\spxentry{fit\_predict()}\spxextra{función incorporada}}

\begin{fulllineitems}
\pysiglinewithargsret{\sphinxbfcode{\sphinxupquote{fit\_predict}}}{\emph{data}}{}
Método para aplicar el algoritmo PINN-LOF a una matriz de datos.
\begin{quote}\begin{description}
\item[{Parámetros}] \leavevmode
\sphinxstyleliteralstrong{\sphinxupquote{numpy.array}} \textendash{} data, matriz de datos

\item[{Tipo del valor devuelto}] \leavevmode
numpy.array de puntuaciones de anomalía

\end{description}\end{quote}

\end{fulllineitems}



\chapter{OUTRES}
\label{\detokenize{index:outres}}
Outres es un algoritmo que propone desarrollar una puntuación de anomalías basada
en la desviación de objetos en las proyecciones subespaciales. Para la selección de
dichos subespacios se analiza la uniformidad de los datos en ellos.
\index{OUTRES() (función incorporada)@\spxentry{OUTRES()}\spxextra{función incorporada}}

\begin{fulllineitems}
\phantomsection\label{\detokenize{index:OUTRES}}\pysiglinewithargsret{\sphinxbfcode{\sphinxupquote{OUTRES}}}{\emph{epsilon=15}, \emph{alpha=0.01}}{}
Constructor para la creación del objeto de la clase OUTRES.
\begin{quote}\begin{description}
\item[{Parámetros}] \leavevmode\begin{itemize}
\item {} 
\sphinxstyleliteralstrong{\sphinxupquote{int}} \textendash{} epsilon, radio para la selección del vecindario

\item {} 
\sphinxstyleliteralstrong{\sphinxupquote{float}} \textendash{} alpha, limite de uniformidad que se permite como interesante

\end{itemize}

\item[{Tipo del valor devuelto}] \leavevmode
objeto de la clase OUTRES

\end{description}\end{quote}

\end{fulllineitems}

\index{fit\_predict() (función incorporada)@\spxentry{fit\_predict()}\spxextra{función incorporada}}

\begin{fulllineitems}
\pysiglinewithargsret{\sphinxbfcode{\sphinxupquote{fit\_predict}}}{\emph{data}}{}
Método para aplicar el algoritmo OUTRES a una matriz de datos.
\begin{quote}\begin{description}
\item[{Parámetros}] \leavevmode
\sphinxstyleliteralstrong{\sphinxupquote{numpy.array}} \textendash{} data, matriz de datos

\item[{Tipo del valor devuelto}] \leavevmode
numpy.array de puntuaciones de anomalía

\end{description}\end{quote}

\end{fulllineitems}



\chapter{ODIN}
\label{\detokenize{index:odin}}
Outlier Detection using Indegree Number (ODIN),es un algoritmo que hace
uso del grafico de los k-vecinos más cercanos y usa el grado de los nodos
para el calculo de anomalías
\index{ODIN() (función incorporada)@\spxentry{ODIN()}\spxextra{función incorporada}}

\begin{fulllineitems}
\phantomsection\label{\detokenize{index:ODIN}}\pysiglinewithargsret{\sphinxbfcode{\sphinxupquote{ODIN}}}{\emph{k=20}, \emph{t=0.01}}{}
Constructor para la creación del objeto de la clase ODIN.
\begin{quote}\begin{description}
\item[{Parámetros}] \leavevmode\begin{itemize}
\item {} 
\sphinxstyleliteralstrong{\sphinxupquote{int}} \textendash{} k, número de k vecinos a calcular

\item {} 
\sphinxstyleliteralstrong{\sphinxupquote{int}} \textendash{} t, umbral de dicisión

\end{itemize}

\item[{Tipo del valor devuelto}] \leavevmode
objeto de la clase ODIN

\end{description}\end{quote}

\end{fulllineitems}

\index{fit\_predict() (función incorporada)@\spxentry{fit\_predict()}\spxextra{función incorporada}}

\begin{fulllineitems}
\pysiglinewithargsret{\sphinxbfcode{\sphinxupquote{fit\_predict}}}{\emph{data}}{}
Método para aplicar el algoritmo ODIN a una matriz de datos.
\begin{quote}\begin{description}
\item[{Parámetros}] \leavevmode
\sphinxstyleliteralstrong{\sphinxupquote{numpy.array}} \textendash{} data, matriz de datos

\item[{Tipo del valor devuelto}] \leavevmode
numpy.array de decisión 1-0

\end{description}\end{quote}

\end{fulllineitems}



\chapter{MeanDIST}
\label{\detokenize{index:meandist}}
El algoritmo MeanDIST usa la la media de las distancias en su vecindario
para ordenar a los vérticesy seleccionar los que más se desvian.
\index{MeanDIST() (función incorporada)@\spxentry{MeanDIST()}\spxextra{función incorporada}}

\begin{fulllineitems}
\phantomsection\label{\detokenize{index:MeanDIST}}\pysiglinewithargsret{\sphinxbfcode{\sphinxupquote{MeanDIST}}}{\emph{k=20}, \emph{t=1.5}}{}
Constructor para la creación del objeto de la clase MeanDIST.
\begin{quote}\begin{description}
\item[{Parámetros}] \leavevmode\begin{itemize}
\item {} 
\sphinxstyleliteralstrong{\sphinxupquote{int}} \textendash{} k, número de k vecinos a calcular

\item {} 
\sphinxstyleliteralstrong{\sphinxupquote{int}} \textendash{} t, parámatro para ampliar o reducir el umbral.

\end{itemize}

\item[{Tipo del valor devuelto}] \leavevmode
objeto de la clase MeanDIST

\end{description}\end{quote}

\end{fulllineitems}

\index{fit\_predict() (función incorporada)@\spxentry{fit\_predict()}\spxextra{función incorporada}}

\begin{fulllineitems}
\pysiglinewithargsret{\sphinxbfcode{\sphinxupquote{fit\_predict}}}{\emph{data}}{}
Método para aplicar el algoritmo MeanDIST a una matriz de datos.
\begin{quote}\begin{description}
\item[{Parámetros}] \leavevmode
\sphinxstyleliteralstrong{\sphinxupquote{numpy.array}} \textendash{} data, matriz de datos

\item[{Tipo del valor devuelto}] \leavevmode
numpy.array de decisión 1-0

\end{description}\end{quote}

\end{fulllineitems}



\chapter{KDIST}
\label{\detokenize{index:kdist}}
El algoritmo KDIST el máximo de las distancias a
sus k-vecinos más cercanos para ordenar a los vértices y
seleccionar los que más se desvian.
\index{KDIST() (función incorporada)@\spxentry{KDIST()}\spxextra{función incorporada}}

\begin{fulllineitems}
\phantomsection\label{\detokenize{index:KDIST}}\pysiglinewithargsret{\sphinxbfcode{\sphinxupquote{KDIST}}}{\emph{k=20}, \emph{t=1.5}}{}
Constructor para la creación del objeto de la clase KDIST.
\begin{quote}\begin{description}
\item[{Parámetros}] \leavevmode\begin{itemize}
\item {} 
\sphinxstyleliteralstrong{\sphinxupquote{int}} \textendash{} k, número de k vecinos a calcular

\item {} 
\sphinxstyleliteralstrong{\sphinxupquote{int}} \textendash{} t, parámatro para ampliar o reducir el umbral.

\end{itemize}

\item[{Tipo del valor devuelto}] \leavevmode
objeto de la clase KDIST

\end{description}\end{quote}

\end{fulllineitems}

\index{fit\_predict() (función incorporada)@\spxentry{fit\_predict()}\spxextra{función incorporada}}

\begin{fulllineitems}
\pysiglinewithargsret{\sphinxbfcode{\sphinxupquote{fit\_predict}}}{\emph{data}}{}
Método para aplicar el algoritmo KDIST a una matriz de datos.
\begin{quote}\begin{description}
\item[{Parámetros}] \leavevmode
\sphinxstyleliteralstrong{\sphinxupquote{numpy.array}} \textendash{} data, matriz de datos

\item[{Tipo del valor devuelto}] \leavevmode
numpy.array de decisión 1-0

\end{description}\end{quote}

\end{fulllineitems}

\begin{itemize}
\item {} 
\DUrole{xref,std,std-ref}{genindex}

\item {} 
\DUrole{xref,std,std-ref}{modindex}

\item {} 
\DUrole{xref,std,std-ref}{search}

\end{itemize}



\renewcommand{\indexname}{Índice}
\printindex
\end{document}